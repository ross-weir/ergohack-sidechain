\documentclass{article}   % list options between brackets

\usepackage{color}
\usepackage{graphicx}
%% The amssymb package provides various useful mathematical symbols
\usepackage{amssymb}
%% The amsthm package provides extended theorem environments
%\usepackage{amsthm}
\usepackage{amsmath}

\usepackage{listings}

\usepackage{hyperref}

\usepackage{systeme}

\def\shownotes{1}
\def\notesinmargins{0}

\ifnum\shownotes=1
\ifnum\notesinmargins=1
\newcommand{\authnote}[2]{\marginpar{\parbox{\marginparwidth}{\tiny %
  \textsf{#1 {\textcolor{blue}{notes: #2}}}}}%
  \textcolor{blue}{\textbf{\dag}}}
\else
\newcommand{\authnote}[2]{
  \textsf{#1 \textcolor{blue}{: #2}}}
\fi
\else
\newcommand{\authnote}[2]{}
\fi

\newcommand{\knote}[1]{{\authnote{\textcolor{green}{kushti notes}}{#1}}}
\newcommand{\mnote}[1]{{\authnote{\textcolor{red}{scalahub notes}}{#1}}}

\usepackage[dvipsnames]{xcolor}
\usepackage[colorinlistoftodos,prependcaption,textsize=tiny]{todonotes}


% type user-defined commands here
\usepackage[T1]{fontenc}

\usepackage{flushend}


\newcommand{\ma}{\mathcal{A}}
\newcommand{\mb}{\mathcal{B}}
\newcommand{\he}{\hat{e}}
\newcommand{\sr}{\stackrel}
\newcommand{\ra}{\rightarrow}
\newcommand{\la}{\leftarrow}
\newcommand{\state}{state}

\newcommand{\ignore}[1]{} 
\newcommand{\full}[1]{}
\newcommand{\notfull}[1]{#1}
\newcommand{\rand}{\stackrel{R}{\leftarrow}}
\newcommand{\mypar}[1]{\smallskip\noindent\textbf{#1.}}

\begin{document}

\title{Two-Way Pegged Sidechains On Ergo (ErgoHack Report)}
\author{kushti, soysor}


\maketitle

\begin{abstract}
In this report we outline some ways to implement pegged sidechains on Ergo with smart contract powered transfers, where both main and
  side chains are based on UTXOs and support ErgoTree (in its current form, at the same time, sidechain may support more features).
In such setting, transfer from sidechain and back can be done via on-chain contract, then security issues are reduced to
  sidechain consensus issue only.
\end{abstract}


\section{Introduction}

A sidechain~\cite{back2014enabling} is a secondary blockchain connected to the main blockchain with a two-way peg.
Pegged sidechains enable ERG (native cryptocurrency of the Ergo blockchain) and other Ergo blockchain assets to be
transferred between multiple Ergo-like blockchains. Sidechains can be considered as playgrounds for experimental features,
or scalability solution~(for example, a sidechain may support shielded transactions or other privacy-preserving
techniques), allowing to off-load transaction from the main chain to sidechains.

\subsection{Ergo blockchain details}

We provide relevant Ergo blockchain needed here. Prior knowledge of UTXO accounting model, Bitcoin, data structures etc
is needed.
\begin{itemize}
\item{} Ergo has UTXO set commitments, so set of unspent outputs after processing transactions in a block is committed via
authenticated AVL+ tree, with tree digest included into a block header.
\item{} Last headers as well as some fields from upcoming header can be read from a contract.
\end{itemize}

\section{Sidechain Data on the Mainnet}

In this section we consider how sidechain data can be stored on the main-chain.

A sidechain progress can be represented in the simplest form as a tuple $(h, T_h, U_h, C_h)$, where
$h$ is a sidechain height, $T_h$ are state changes (transactions) done in $h$, $U_h$ is digest of AVL+ tree built on top
of UTXO set after processing the changes $T_h$, $C_h$ is AVL+ tree which contains all the previous sidechain states in form
of key-value pairs $h -> hash(h || T_h || U_{h-1})$.

First of all, we need to write this minimal data somewhere, to read in contracts later. Currently, we can think about two options:

\begin{itemize}
\item{} there's extension section of a block, which contains key -> value pairs, and root of a Merkle tree built upon the pairs is included into a
block header (and can be read by a contract). We can include different sidechains here, or even a tree built on top of sidechains. The only issue with this approach is that Merkle trees are not supported in ErgoTree/Script yet

\item{} we can just put sidechain data into a box identified by some NFT
\end{itemize}

\subsection{Sidechain Consensus}


There could be different option how sidechain can make a progress
\begin{itemize}
\item{} merged mining - PoW fields (nonce, solution) to be added to the tuple, and writted on the Ergo blockchain. Then sidechain header would be Ergo block header + extension Merkle path or path to box in UTXO set tree + sidechain tuple with PoW data. Security: subset of Ergo miners (so honest majority amongst them)

\item{} if sidechain data is in a box, we can make the box spendable for an Ergo block miner (by using miner pubkey from context like done in emission contract). Security: subset of Ergo miners (so honest majority amongst them). There could be other options here (multisig etc)
\end{itemize}

\section{Transfers}

\subsection{Mainchain to Sidechain}

\subsection{Sidechain to Mainchain}

\section{Security}

\section{Evaluation}

\section{Further Work}



\newpage
\bibliography{sources}
\bibliographystyle{ieeetr} 

\end{document}
